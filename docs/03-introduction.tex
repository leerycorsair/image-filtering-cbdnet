\chapter*{ВВЕДЕНИЕ}
\addcontentsline{toc}{chapter}{ВВЕДЕНИЕ}

Современные технологии позволяют нам использовать различные устройства для регистрации изображений практически повсюду: смартфоны с фронтальными и тыльными камерами, видеокамеры в автомобилях, на улицах, общественном транспорте, жилых домах, магазинах \cite{cameras}. Однако, изображения, получаемые в реальном мире, часто сопровождаются шумами, вызванными различными факторами, такими как нестабильность электрического сигнала, неисправности датчиков камеры, плохие условия освещения и ошибки при передаче данных. Эти шумы негативно влияют на качество захваченных изображений, ведь они приводят к потере информации, замещая исходные значения пикселей случайными значениями.

Одной из фундаментальных задач в области обработки изображений является эффективное удаление шума, при этом сохранив важные детали, необходимые для последующего распознавания. Разработка решений, способных удалить шумы из изображений, является неотъемлемой частью работы в области компьютерного зрения и обработки изображений. Цель состоит в улучшении качества изображений и сохранении в них наибольшего количества информации.

Таким образом, в настоящее время существует необходимость в разработке инновационных методов и техник, которые могут автоматически удалить шумы из захваченных изображений, вне зависимости от причин их возникновения. Это поможет повысить качество изображений, обеспечивая более четкую и точную информацию для анализа и распознавания объектов, и даст возможность более эффективного использования изображений в различных областях, таких как компьютерное зрение, медицинская диагностика, видеонаблюдение и другие.

Целью данной работы является разработка метода фильтрации малоразмерных шумов на цветных изображениях с помощью сверточных нейронных сетей.

Для достижения указанной выше цели следует выполнить следующие задачи:
	\begin{itemize}
		\item ввести основные понятия предметной области фильтрации малоразмерных шумов на изображениях;
		\item рассмотреть существующие методы фильтрации малоразмерных шумов на изображениях и сравнить их;
  	\item разработать метод фильтрации малоразмерных шумов на цветных изображениях с помощью сверточных нейронных сетей;
		\item разработать программный комплекс для взаимодействия с разработанным методом;
        \item провести исследование эффективности, применимости разработанного программного обеспечения и выполнить сравнение результатов работы реализованного метода с результатами, полученными с помощью известных методов.
	\end{itemize}
