\chapter*{ЗАКЛЮЧЕНИЕ}
\addcontentsline{toc}{chapter}{ЗАКЛЮЧЕНИЕ} 

В результате выполнения дипломной работы был проведен анализ предметной области: были рассмотрены общие сведения о построении изображений, виды помех, которые могут возникать. Были рассмотрены алгоритмы нейронных сетей в следующих аспектах:
\begin{itemize}
    \item общие принципы работы нейронных сетей;
    \item различные функции активации и функции потерь;
    \item особенности обучения нейронных сетей, метод обратного распространения ошибки, а также критерии выбора данных для обучения;
    \item виды нейронных сетей.
\end{itemize}

Был проведен анализ методов обработки изображений и была составлена таблица сравнения по нескольким критериям.

Был спроектирован и описан метод фильтрации малоразмерных шумов на цветных изображениях с помощью сверточных нейронных сетей, приведены соответствующие схемы и IDEF0--диаграммы, также выполнено описание выбранного набора данных для обучения нейронной сети. 

Были описаны выбранные средства разработки, реализован программный комплекс, описан формат входных и выходных данных.

Было проведено исследование эффективности и применимости разработанного программного обеспечения, а также выполнено сравнение результатов работы реализованного метода с результатами, полученными с помощью известных методов. Выявлены достоинства (работа с реальными шумами и высокая степень сохранения деталей изображения) и недостатки (при работе с синтетическими шумами наблюдается искажение цветов и в форматах изображений, использующих сжатие для хранения, наблюдается ухудшение качества работы) разработанного программного комплекса

Таким образом, поставленная цель работы --- разработка метода фильтрации малоразмерных шумов на цветных изображениях с помощью сверточных нейронных сетей, была выполнена в полном объеме.